\paragraph{Question 3.1}
Si on compare deux s\'equences de longeurs $d$ identiques (pire cas),
les algorithmes des parties pr\'ec\'edentes ont besoins \`a une
constante pr\^et (1 octet) de $d^2$ espace m\'emoire que l'on note
$T_{mem}$, soit $d=\sqrt{T_{mem}}$. Le tableau ci-dessous nous donne
quelques applications num\'eriques.
\begin{table*}[h]
  \centering
  \begin{tabular}{rc|c|c}
    $T_{mem}=$&8Go&16Go&32Go\\
    \hline
    $d\approx$&89K&126K&179K
  \end{tabular}
\end{table*}

\paragraph{Question 3.2}
On convient que l'on dispose d'un algorithme \verb'P(i,j)', cachant
une structure de donn\'ees sp\'ecifique, donnant les p\'enalit\'es de
correspondances $\delta_{x_iy_j}$ tel qu'en complexit\'e spatiale, il
ne d\'epend que de la taille de la taille des alphabets utilis\'es
pour coder les s\'equences $X=(x_i)_{0\leq i\leq m}$ et
$Y=(y_j)_{0\leq j\leq n}$ que l'on souhaite comparer. Entre autre on
ne stocke pas les p\'enalit\'es de correspondances nulles.\\\\
Pour \verb'COUT2', on est toujours sur une approche type programmation
dynamique : le calcul de $F(i,j)$ par l'algorithme auxiliaire
\verb'MEMO-COUT2' ne se fait que s'il n'a pas d\'ej\`a \'et\'e fait,
l'argument \verb'l' qui est pass\'ee en argument indique laquelle des
deux lignes $i-1$ (\verb'l=0') ou $i$ (\verb'l=1' par exemple) est
n\'ecessaire pour le calcul de $F(i,j)$.\\\\
L'algorithme auxiliaire \verb'MAJ-MEMO', auqel on passe en argument
des r\'ef\'erences aux tableaux de m\'emo\"isation, fait la transition
avec la paire de lignes suivante.
\begin{verbatim}
MAJ-MEMO(T0, T1, i)
  T0 <- T1
  T1[0] <- i * d-gap
  POUR j = 1..n FAIRE
    T1[j] = -1

MEMO-COUT2(T0, T1, l, i, j)
  SI l = 0 ALORS RETOURNER T0[j]
  SI T1[j] < 0 ALORS
    RETOURNER MIN(MEMO-COUT2(T0, T1, 0, i, j-1) + P(i,j),
                  MEMO-COUT2(T0, T1, 1, i, j-1) + d-gap,
                  MEMO-COUT2(T0, T1, 0, i, j) + d-gap)
  SINON RETOURNER T1[j]

COUT2(i, j)
  T0[0..j] : tableau
  T1[0..j] : tableau
  T0[0] = 0
  T1[0] = d-gap
  POUR l = 1..j FAIRE
    T0[l] = l * d-gap
  POUR l = 1..i FAIRE
    T1[l] = -1
  POUR k = 1..i FAIRE
    POUR l = 1..j FAIRE
      MEMO-COUT2(T0, T1, 1, k, l)
    MAJ-MEMO(T0, T1, k+1)
  RETOURNER T1[j]
\end{verbatim}
La complexit\'e temporelle de \verb'COUT2(i,j)' est la m\^eme que
celle de \verb'COUT1(i,j)' \emph{i.e.} $\Theta(ij)$. Cependant, si on
note $cs(P(i,j))$ la complexit\'e spatiale de \verb'P', la
complexit\'e spatiale de \verb'COUT2(i,j)' est en
$\Theta(j+cs(P(i,j)))$. En supposant que $cs(P(i,j))$ est
asymptotiquement major\'ee par $\sigma^2$ avec $\sigma$ la taille de
l'alphabet utilis\'e pour encoder nos s\'equences, et que $\sigma^2$
est major\'e par $j$ ce qui est une hypoth\'ese convenable avec les
hypoth\'ese faites sur la complexit\'e spatiale de \verb'P' et la
taille des s\'equences \`a compar\'es, \verb'COUT2(i,j)' est en $O(j)$.

\paragraph{Question 3.3}
\verb'COUT2' suit le m\^eme principe que \verb'COUT1' en translatant
le probl\`eme au sous graphe induit par le sous-ensemble de sommets
$\big\{(k,l)\ \big|\ i\leq k\leq m\ \text{et}\ j\leq l\leq n\big\}$.
\begin{verbatim}
COUT2BIS(i, j, m, n)
  T0[0..n-j] : tableau
  T1[0..n-j] : tableau
  T0[0] <- 0
  T1[0] <- d-gap
  POUR l = 1..n-j FAIRE
    T1[l] <- l * d-gap
  POUR k = 1..m-i FAIRE
    POUR l = 1..n-j FAIRE
      MEMO-COUT2(T0, T1, 1, k, l)
    MAJ-MEMO(T0, T1, k)
  RETOURNER T1[n-j]
\end{verbatim}
\paragraph{Notations}
\begin{itemize}
\item Pour $1\leq i\leq m-1$ et $1\leq j\leq n-1$ et deux s\'equences
  $X$ et $Y$, on note $G_{ij}$ (resp. $H_{ij}$) le sous-graphes de
  $G_{XY}$ induit par le sous-ensemble de sommets
  $\big\{(k,l)\ |\ k\leq i\ \text{et}\ l\leq j\big\}$
  (resp. $\big\{(k,l)\ |\ k\geq i\ \text{et}\ l\geq j\big\}$).
  On remarque que l'on a $G_{ij}\cap H_{ij}=\big(\{(i,j)\},\{\}\big)$.
\item Pour un graphe $G=(S,A)$ orient\'e, valu\'e et $s_o,s\in S$, on
  note $d_{G_{s_0}}(s)$ le co\^ut du plus court chemin de $s_o$ \`a
  $s$, et $c_G$ la fonction valuation de $G$.
\item Pour simplifier les notations on pose
  $d_{G_{XY}}=d_{{G_{XY}}_{(0,0)}}$, $d_{G_{ij}}=d_{{G_{ij}}_{(0,0)}}$
  et $d_{H_{ij}}=d_{{H_{ij}}_{(i,j)}}$. On remarque que
  $g(i,j)=d_{G_{i,j}}\big((i,j)\big)$ et
  $h(i,j)=d_{H_{ij}}\big((m,n)\big)$
\end{itemize}

\paragraph{Question 3.4}
Les cas ou $(i,j)=(0,0)$ ou $(m,n)$ sont triviaux. Soit
$1\leq i\leq m-1$ et $1\leq j\leq n-1$ tels que qu'un plus court
chemin de $(0,0)$ \`a $(m,n)$ dans $G_{XY}$ passe par le sommet
$(i,j)$. Il existe $(s_{g_0}=(0,0),\ldots,s_{g_{l'}}=(i,j))$ un plus
court chemin de $(0,0)$ \`a $(i,j)$ dans $G_{ij}$ et
$(s_{h_0}=(i,j),\ldots,s_{h_l}=(m,n))$ un plus court chemin de $(i,j)$
\`a $(m,n)$ dans $H_{ij}$.

\begin{figure*}[!h]
\centering
\begin{tikzpicture}[scale=0.9]
  \SetGraphUnit{2}
  \Vertex[L=${i,j}$]{C}
  \NO[L=$s_2$](C){N}
  \NOWE[L=$s_1$](C){NW}
  \WE[L=$s_3$](C){W}
  \SO[L=$s_2'$](C){S}
  \SOEA[L=$s_1'$](C){SE}
  \EA[L=$s_3'$](C){E}
  \tikzset{EdgeStyle/.style = {->}}
  \Edge[label=$\delta_{gap}$](N)(C)
  \Edge[label=$\delta_{gap}$](W)(C)
  \Edge[label=$\delta_{gap}$](C)(E)
  \Edge[label=$\delta_{gap}$](C)(S)
  \Edge[label=$\delta$](NW)(C)
  \Edge[label=$\delta'$](C)(SE)
  \draw[draw=blue] (-0.6,-3) -- (-0.6,0.6) -- (3,0.6);
  \draw[blue] (2.5,1) node {$H_{ij}$};
  \draw[draw=orange] (-3,-0.6) -- (0.6,-0.6) -- (0.6,3);
  \draw[orange] (-2.5,-1) node {$G_{ij}$};
\end{tikzpicture}
\end{figure*}
%%% Local Variables:
%%% mode: latex
%%% TeX-master: "../rapport"
%%% End:
 

Les sommets $s_1,s_2,s_3,$ (resp. $s_1',s_2',s_3'$) repr\'esent\'es
sur la figure ci-dessus sont les seuls pr\'edecesseurs
(resp. succeseurs) de $(i,j)$ dans $G_{XY}$. Si on note
$((0,0),\ldots,s,(i,j),s',\ldots,(m,n))$ un plus court chemin de
$(0,0)$ \`a $(m,n)$ passant par $(i,j)$ dans $G_{XY}$, on a
n\'ecessairement $s\in\{s_1,s_2,s_3\}$ et $s'\in\{s_1',s_2',s_3'\}$.
\nopagebreak

On peut alors -- en utilisant les notations introduites -- facilement
montrer par r\'ecurrence finie sur $k\in\{0\ldots l\}$ l'\'equation
(1). Les \'equations (2) et (3) se d\'eduisent alors simplement en
interpr\'etant les notations.
\begin{align}
  d_{G_{XY}}\big(m,n\big)
  &=d_{G_{ij}}(i,j)+\sum_{k=0}^{l-1}c_{H_{ij}}(s_{h_k},s_{h_{k+1}})\\
  &=d_{G_{ij}}(i,j)+d_{H_{ij}}(m,n)\\
  &=g(i,j)+h(i,j)
\end{align}

\paragraph{Question 3.5}
Listons les complexit\'es spatiales de l'ensemble des structures de
donn\'ees misent en jeu par \verb'SOL2(0,0,m,n)'.\\
\begin{itemize}
\item Structures intras\'eques :
\begin{table*}[h]
  \centering
\begin{tabular}{l|c}
  %structure&compl\'exit\'e\\
  %\hline
  $X2a,Y2b$&$\Theta(1)$\\
  $Y2a,F2a$&$\Theta(n)$\\
  $X2b,F2b$&$\Theta(m)$\\
\end{tabular}
\end{table*}
\item Les appels de \verb'SOL1' sur $X2a$ et $Y2a$ avec $F2a$ ont une
  complexit\'e spatiale en $\Theta(n)$.
\item Les appels de \verb'SOL1' sur $X2b$ et $Y2b$ avec $F2b$ ont une
  complexit\'e spatiale en $\Theta(m)$.
\item Un pire appel en termes de m\'emoire de \verb'COUT2' se fait
  pour $i=0$ et $j=\lceil\frac{m-n}{2}\rceil$. Soit une compl\'exit\'e
  spatiale en $O\big(max(m,n)\big)$ (\'etape 1).
\item De m\^eme pour la complexit\'e spatiale de \verb'COUT2BIS'.
\end{itemize}
\paragraph{}
La complexit\'e spatiale de l'appel \verb'SOL2(0,0,m,n)' est donc bien
en $O(m+n)$.
%%% Local Variables:
%%% mode: latex
%%% TeX-master: "rapport"
%%% End:
