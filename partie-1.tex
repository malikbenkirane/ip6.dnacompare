\paragraph{Question 1.1}
On peut \'enum\'erer tous les les sous-ensembles $C_k$ de paires
$(x_i,y_j)$ pour deux s\'equences $X=(x_i)_{1\leq i\leq d}$ et
$Y=(y_i)_{1\leq i\leq d}$ et tester si se sont des alignements. Il y a
$d^2$ paires possibles, soit la paires est dans $C_k$, soit elle ne
l'est pas, ce qui nous donne $2^{d^2}$ sous-ensembles $C_k$ à tester.
\paragraph{Question 1.2}
Soit $M$ un alignement de $X$ et $Y$. Supposons que $(x_m,y_n)\not\in M$
et qu'il existe $i\leq m, j\leq n$ tels que $(x_i,y_n)\in M$ et
$(x_m,y_j)\in M$ (i.e. $x_m$ et $y_n$ apparaissent dans
$M$). N\'ecessairement $i<m$ et $j<n$ puisque $x_m$ et $y_n$
apparaissent au plus une fois. De plus comme il n'y a pas de
croisement dans $M$ et que $i<m$, $n<j$. Contradiction.
\paragraph{Notation}
Par abus, pour un alignement $M$ de deux s\'equences 
$X=(x_i)_{0\leq i\leq m}$ et $Y=(x_j)_{0\leq j\leq n}$,
on notera $x_k\in M$ lorsqu'il existe $j$, tel que $(x_k,y_j)\in M$,
sym\'etriquement on pourra aussi \'ecrire $y_l\in M$, et on d\'eduit les
n\'egations respectives.
\paragraph{Question 1.3}
Les trois cas de figures suivants se d\'eduisent de la r\'eflexion
pr\'ec\'edente :
\begin{itemize}
\item $x_m\not\in M$
\item $y_n\not\in M$
\item $(x_m,y_n)\in M$
\end{itemize}
\paragraph{Notation}
Pour un alignement $M$ de deux s\'equences $X=(x_i)_{1\leq i\leq m}$
et $Y=(y_i)_{1\leq i\leq n}$. On note $M_{i,j}$ le sous-alignement
$\Big\{(x_k,y_l)\in M\ |\ k\leq i,\ l\leq j\Big\}$. On convient que
$M=M_{m,n}$. Il est \'evident que $M_{i,j}$ est un alignement de ses
sous-s\'equences correspondantes i.e. $(x_k)_{k\leq i}$ et
$(y_l)_{l\leq j}$.
\paragraph{Question 1.4}
Consid\'erons $M^*$ un alignement de co\^ut minimal des s\'equences
$(x_i)_{1\leq i\leq m}$ et $(y_j)_{1\leq j\leq n}$. Un tel alignement
existe puisque l'ensemble des alignements pour une s\'equence donn\'ee
est fini.
\begin{enumerate}
\item lorsque $\underline{(x_m,y_n)\in M^*}$
  \begin{equation*}
    F(m,n) = f(M^*) = \underbrace{
      \sum_{(x_i,y_j)\in M_{m-1,n-1}^*}\delta_{x_iy_j} +
      \sum_{x_i\not\in M_{m-1,n-1}^*}\delta_{gap} + \sum_{y_j\not\in M_{m-1,n-1}^*}\delta_{gap}}_{f(M_{m-1,n-1}^*)}
    + \delta_{x_my_n}.
  \end{equation*}
  Or $M^*_{m-1,n-1}$ est de co\^ut minimal pour ses sous-s\'equences
  coresspondantes. Donc :
  \begin{equation*}
    F(m,n) = F(m-1, n-1) + \delta_{x_my_n}.
  \end{equation*}
\item lorsque $\underline{x_m\not\in M^*}$, $M^*_{m-1,n}$ est optimal pour ses
  sous-s\'equences correspondantes. Donc :
  \begin{equation*}
    F(m,n)=F(m-1,n)+\delta_{gap}
  \end{equation*}
\item lorsque $\underline{y_n\not\in M^*}$, de m\^eme $M^*_{m,n-1}$
  est optimal et :
  \begin{equation*}
    F(m,n)=F(m,n-1)+\delta_{gap}
  \end{equation*}
\end{enumerate}
\paragraph{Question 1.5}
Afin de d'optimiser le co\^ut d'un alignement il suffit de prendre la
plus petite valeur des trois cas de figures. Soit pour 
$i\geq 1,\ j\geq 1$ :
\begin{equation*}
  F(i,j)=min\Big\{F(i-1,j-1)+\delta_{x_iy_j},
  \ F(i-1,j)+\delta_{gap},\ F(i,j-1)+\delta_{gap}\Big\}
\end{equation*}
\paragraph{Question 1.6}
Soit $i\in \{1\ldots m\}$. Tout alignement $M_{i,0}$ est vide. Donc,
\begin{equation*}
  F(i,0) = 
  \underbrace{
    \sum_{(x_i,y_j)\in M_{i,0}^*}\delta_{x_iy_j} +
    \sum_{y_j\not\in M_{i,0}^*}\delta_{gap}
  }_{=0} +
  \sum_{x_i\not\in M_{i,0}^*}\delta_{gap}
  = i\delta_{gap}.
\end{equation*}
Par sym\'etrie, $F(0,j)=j\delta_{gap}$ pour tout $j\in\{1\ldots n\}$.
\paragraph{Question 1.7}
On convient que l'on dispose de la primitive \verb'MIN' renvoyant la
valeur minimale d'un n-uplet. On convient aussi d'une primitive
\verb'creerMatrice(N,M,V)' qui initialise une matrice de taille
$N\times M$ \`a valeur unique \verb'V' et dont les indicices sont
compris entre \verb'[0,0]' et \verb'[N-1,M-1]'.
\\\\
L'approche, ici, est de type programmation dynamique :
\verb'MEMO-COUT1' ne calcul \verb'C[i,j]' ($F(i,j)$) que s'il n'a pas
\'et\'e calcul\'e pr\'ec\'edemment. \verb'COUT1' appel
\verb'MEMO-COUT1' $m\times n$ fois soit une compl\'exit\'e temporel en
$\Theta(mn)$. Cette m\'emo\"isation nous donne une complexit\'e
spatiale en $\Theta(mn)$.
\\\\
La matrice \verb'P' correspond aux p\'enalit\'es de correspondances
$\delta_{x_iy_j}$ et \verb'd-gap' \`a la p\'enalit\'e $\delta_{gap}$.
\begin{verbatim}
MEMO-COUT1(C : Matrice(m,n), P : Matrice(m,n), d-gap, i, j)
  SI C[i,j] < 0 ALORS
    RETOURNER MIN(MEMO-COUT1(C, i-1, j-1) + P[i,j],
                  MEMO-COUT1(C, i-1, j) + d-gap,
                  MEMO-COUT1(C, i, j-1) + d-gap)
  SINON RETOURNER C[i,j]

COUT1(P : Matrice(m,n), d-gap)
  C <- creerMatrice(m+1,n+1,-1)
  C[0,0] <- 0
  POUR i = 1..m FAIRE
    C[i,0] <- i * d-gap
  POUR j = 1..n FAIRE
    C[0,j] <- j * d-gap
  POUR i = 1..m FAIRE
    POUR j = 1..n FAIRE
      C[i,j] = MEMO-COUT1(C, P, d-gap, i, j)
  RETOURNER C[m,n]
\end{verbatim}
\pagebreak
\paragraph{Question 1.8}
On dispose des \verb'C[i,j]' : matrice gloable des co\^uts des sous-alignements
optimaux $M_{i,j}^*$.
%Le tableau \verb'X' (resp. \verb'Y') correspond \`a la s\'equence 
%$(x_i)_{1\leq i\leq m}$ (resp. $(y_j)_{1\leq j\leq n}$).
La liste de paire \verb'M', intialis\'ee \`a la liste vide au d\'ebut
de l'algorithme est construite r\'ecursivement par
\verb'REC-SOL1'. Elle correspond \`a un alignement optimal pour la
fonction co\^ut $f$.\\\\
\verb'P' et \verb'd-gap' ont la m\^eme s\'emantique qu'\`a la question
1.7.
\begin{verbatim}
REC-SOL1(M : liste de paires, i, j, P, d-gap)
  SI i = 0 OU j = 0 ALORS TERMINER
  SI C[i,j] = C[i-1,j-1] + P[i,j] ALORS
    M.append((i,j))
    REC-SOL1(M, i-1, j-1, P, d-gap)
  SINON SI C[i,j] = C[i-1,j] + d-gap ALORS
    REC-SOL1(M, i-1, j, P, d-gap)
  SINON REC-SOL1(M, i, j-1, P, d-gap)

SOL1(m, n, P, d-gap)
  M <- creerListeVide()
  REC-SOL1(M, m, n, P, d-gap)
  RETOURNER M
\end{verbatim}
\verb'REC-SOL1(m,n)' \`a une complexit\'e temporelle en
$O(\max(m,n))$. Globalement cela nous donne toujours une complexit\'e
en $\Theta(mn)$. La complexit\'e spatiale globale ne varie pas non
plus ($\Theta(mn)$) pusique \verb'M' occupe asymptotiquement au plus
$O(m+n)$ espace.
%%% Local Variables:
%%% mode: latex
%%% TeX-master: "rapport"
%%% End:
